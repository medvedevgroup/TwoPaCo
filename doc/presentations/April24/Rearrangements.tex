\documentclass[svgnames,14pt]{beamer}
\usepackage[utf8]{inputenc}
\usepackage{caption}
\usepackage{graphicx}
\usepackage{xcolor}
\usepackage{wrapfig}
\usepackage{algorithm}
\usepackage{algpseudocode}
\title{The 3D Organization of Chromatin Explains Evolutionary Fragile Genomic Regions by \\ \vspace{12pt} \normalsize{Camille Berthelot, Matthieu Muffato, \\ Judith Abecassis and Hugues Roest Crollius} }
\author{Speaker: Ilia Minkin}
\institute{}
\algnotext{EndFor}
\algnotext{EndIf}
\algnotext{EndWhile}
\setbeamertemplate{footline}[frame number]
\setbeamertemplate{navigation symbols}{}
\setbeamertemplate{caption}[numbered]
\captionsetup[figure]{labelformat=empty}

\begin{document}

\date{24th April 2015}
\maketitle

\begin{frame}
\frametitle{Two types of genome alterations}

1. Small point mutations:

{\vspace{12pt} \Large \color{Blue}
ACTTG\\
A{\color{Red}G}T--\hspace{2.7pt}G
\vspace{12pt}}

\pause
2. Large rearrangements:
\begin{itemize}
\item Inversions
\item Transpositions
\item Fusions
\item ...
\end{itemize}
\end{frame}

\begin{frame}
\frametitle{Genome Rearrangements}
\begin{figure}
	\centering
	\includegraphics[scale = 0.20]{BasicRearr.jpg}
\caption{Source: \textit{Dierssen et al, 2009}}
\end{figure}
\end{frame}

\begin{frame}
\frametitle{Motivation}
Rearrangements:
\begin{itemize}
\item Are a major driving force in evolution
\item Play large role in diseases (e.g. cancer)
\end{itemize}
Known causes:
\begin{itemize}
\item Non-homologous end joining
\item Non-allelic homologous recombination
\item Replication fork stalling
\item ...
\end{itemize}
\end{frame}

\begin{frame}
\frametitle{The Big Question}
Are rearrangements more likely to happen in one parts of a genome than the others?
\pause

\vspace{12pt}
Two hypothesis:
\begin{enumerate}
\item Uniform distribution of breakpoints
\item Breakpoints get reused
\end{enumerate}
\pause 

\vspace{12pt}
{\color{Red} Why are some regions of a genome more "fragile" than the others?}
\end{frame}

\begin{frame}
\frametitle{Methodology}
Assume that genes are "unbreakable"

\vspace{12pt}
Then, how does intergene length affects rearrangement rate?

\vspace{12pt}
If breakpoint density is uniform, number of breakpoints should increase
in proportion to intergene length ---  Poisson distribution

\end{frame}

\begin{frame}
\frametitle{Methodology}
Boreoeutheria: the last common ancestor of primates, rodents, and laurasiatherians

\vspace{12pt}
Null hypothesis: breakpoints are distributed uniformly

\vspace{12pt}
Stages of the study:
\begin{itemize}
\item Reconstruct gene order of Boreoeutheria
\item Identify breakpoints w.r.t human, mouse, dog, cow and horse
\item Do Poisson regression of "breakage rate" 
\item Expect linear law if null hypothesis is true
\end{itemize}
\end{frame}

\begin{frame}
\frametitle{Breakpoint Identification}
\begin{figure}
	\centering
	\includegraphics[scale = .75]{Breakpoint.png}
\end{figure}
\end{frame}

\begin{frame}
\frametitle{How to Explain the Equation?}
$$r = 2.4 10^{-3} \times L ^ {0.38}$$

93\% of variation in breakpoint occurrence is explained by intergene length

\vspace{12pt}
Maybe GC content is the real cause?
\end{frame}

\begin{frame}
\frametitle{Is GC Content The Explanation?}
No.
\vspace{12pt}

\pause
Added GC content in regression -- non-significant coefficient
\end{frame}

\begin{frame}
\frametitle{Are CNEs The Explanation?}
CNEs -- conservative non-coding elements.

Located in the intergenes, may affect rearrangement rate.
Do they?
\vspace{12pt}

\pause
No.
\vspace{12pt}

\pause
Added CNE rate in regression -- improved explanation rate only by 3\%
\end{frame}

\begin{frame}
\frametitle{Inversions within Intergenes}
OK, maybe some breakpoints are more likely than the others.

\vspace{12pt}
We work with gene markers --- see only rearrangements disrupting their order.

\vspace{12pt}
What if there are many missing rearrangements within intergenes?

\vspace{12pt}
We can try to simulate rearrangements and see what happens
\end{frame}

\begin{frame}
\frametitle{Inversions within Intergenes}
Rearrangements have been shown to occur between regions in close 3D proximity in the nucleus

\vspace{12pt}
Contact probability is a good proxy for rearrangement probability

\vspace{12pt}
Simulate and sample breakpoint pairs, choose detectable ones

\vspace{12pt}
Even if we restrict to detectable breakpoints only, simulation confirms the random breakpoint hypothesis
\end{frame}

\begin{frame}
\frametitle{Open Chromatin is the Culprit}
Stick with the simulation -- restrict rearrangements to only \textbf{open chromatin} regions

\vspace{12pt}
Voilà -- simulation coincides with the model!

It implies that chromatin state and proximity of genes may explain fragility of some genomic regions
\end{frame}


\begin{frame}
\frametitle{Conclusion}
It seems that chromatin state and proximity of genes may explain fragility of some genomic regions
\end{frame}

\begin{frame}
\begin{center}
\hfill \huge \\
Thank you!
\end{center}
\end{frame}


\end{document}
